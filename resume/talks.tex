%-------------------------------------------------------------------------------
%	SECTION TITLE
%-------------------------------------------------------------------------------
\cvsection{Talks}


%-------------------------------------------------------------------------------
%	CONTENT
%-------------------------------------------------------------------------------
\begin{cventries}

%---------------------------------------------------------
  \cventry
    {Trainer} % Role
    {IBM Watson Care Insights Reach Initiative} % Event
    {Watson Health} % Organizing Institution
    {Apr. - Sep. 2018} % Date(s)
    {
      \begin{cvitems} % Description(s)
        \item {Watson Care Insights aims to analyze longitudinal patient records and real-time data points through HL7 feeds to provide practitioners insights into the patient’s data.  Watson Care Insights seeks to overcome physician-reported challenges incorporating shared decision making technology into their direct patient interactions, and address these challenges by integrating within the physician's workflow in the electronic health record system.}
      \end{cvitems}
    }

%---------------------------------------------------------
  \cventry
    {Peer Presenter} % Role
    {Content \& Data Analytics Learning Exchange} % Event
    {Watson Health} % Organizing Institution
    {Apr. \& May 2018} % Date(s)
    {
      \begin{cvitems} % Description(s)
        \item {Bayesian Statistics using the Monty Hall problem example}
        \item {Naive Bayesian classification for patient matching}
        \item {Splines regression for population vitals ditributions}
      \end{cvitems}
    }

%---------------------------------------------------------
  \cventry
    {Presenter for the Information Services/Information Technology Network Talks} % Role
    {2018 IRI Fall Networks Conference} % Event
    {Innovation Research Interchange} % Organizing Institution
    {Sep. 2018} % Date(s)
    {
      \begin{cvitems} % Description(s)
%        \item {Cognitive solutions for closing the gaps on flat file curation}
        \item {Cognitive computing can use NLP to extract insights from the unstructured data that holds most of the relevant information, but also it can also perform the necessary transformation accurately at the earliest stages of the data integration pipeline.  Combined with the scalability and speed of the data lake architecture, this and other cognitive tools can effectively close the gaps on some of the most formidable flat file curation challenges.}
      \end{cvitems}
    }

%---------------------------------------------------------
  \cventry
    {Lead Presenter for the METL Group} % Role
    {Cross Team Training Series} % Event
    {Explorys} % Organizing Institution
    {Sep. 2016} % Date(s)
    {
      \begin{cvitems} % Description(s)
%        \item {Ingestion using a Data Lake Framework}
        \item {Seven part series in data ingestion approach using the Hadoop software framework.  Data can be aggregated much more quickly and cost-effectively using a data lake framework than in traditional data warehouses, because of the speed and low-cost of massively parallel computing.  All data is stored in the data lake in its native format until it is needed, and each data element has a metadata tag for easy retrieval.  As a result, responses are ad-hoc rather than predetrmined, and reports based on new requirements can be delivered in days or weeks rather than months or years.}
      \end{cvitems}
    }

%---------------------------------------------------------
\end{cventries}
